\documentclass[a4paper, 11pt]{article}
\usepackage[utf8]{inputenc}
\usepackage[russian]{babel} % Cyrillic letters support

\usepackage{latexsym}
\usepackage[empty]{fullpage}
\usepackage{titlesec}
\usepackage{marvosym}
\usepackage[usenames,dvipsnames]{color}
\usepackage{verbatim}
\usepackage{enumitem}
\usepackage{fontawesome5}
\usepackage{xcolor}
\definecolor{darkblue}{rgb}{0.0, 0.0, 0.6}
\usepackage{hyperref}
\hypersetup{
    colorlinks=true,       % false: boxed links; true: colored links
    linkcolor=darkblue,        % color of internal links (change box color with linkbordercolor)
    citecolor=darkblue,        % color of links to bibliography
    filecolor=darkblue,        % color of file links
    urlcolor=darkblue          % color of external links
}
\usepackage{fancyhdr}
\usepackage{tabularx}
\input{glyphtounicode}



%----------FONT OPTIONS----------
\usepackage[defaultsans]{droidsans}
\renewcommand*\familydefault{\sfdefault}
\usepackage[T1]{fontenc}

\pagestyle{fancy}
\fancyhf{} % clear all header and footer fields
\fancyfoot{}
\renewcommand{\headrulewidth}{0pt}
\renewcommand{\footrulewidth}{0pt}

% Adjust margins
\addtolength{\oddsidemargin}{-0.5in}
\addtolength{\evensidemargin}{-0.5in}
\addtolength{\textwidth}{1in}
\addtolength{\topmargin}{-.8in}
\addtolength{\textheight}{1.0in}
\setlength{\textheight}{100cm}

\urlstyle{same}

\raggedbottom
\raggedright
\setlength{\tabcolsep}{0in}

% Sections formatting
\titleformat{\section}{
  \vspace{-2pt}\scshape\raggedright\large
}{}{0em}{}[\color{black}\titlerule \vspace{-5pt}]

% Ensure that generate pdf is machine readable/ATS parsable
\pdfgentounicode=1

%-------------------------
% Custom commands
\newcommand{\resumeItem}[1]{
  \item\small{
    {#1 \vspace{-2pt}}
  }
}

\newcommand{\resumeSubheading}[4]{
  \vspace{-2pt}\item
    \begin{tabular*}{0.97\textwidth}[t]{l@{\extracolsep{\fill}}r}
      \textbf{#1} & #2 \\
      \textit{\small#3} & \textit{\small #4} \\
    \end{tabular*}\vspace{-7pt}
}

\newcommand{\resumeSubSubheading}[2]{
    \item
    \begin{tabular*}{0.97\textwidth}{l@{\extracolsep{\fill}}r}
      \textit{\small#1} & \textit{\small #2} \\
    \end{tabular*}\vspace{-7pt}
}

\newcommand{\resumeProjectHeading}[2]{
    \item
    \begin{tabular*}{0.97\textwidth}{l@{\extracolsep{\fill}}r}
      \small#1 & #2 \\
    \end{tabular*}\vspace{-6pt}
}


\newcommand{\resumeSubItem}[1]{\resumeItem{#1}\vspace{-4pt}}

\renewcommand\labelitemii{$\vcenter{\hbox{\tiny$\bullet$}}$}

\newcommand{\resumeSubHeadingListStart}{\begin{itemize}[leftmargin=0.15in, label={}]}
\newcommand{\resumeSubHeadingListEnd}{\end{itemize}}
\newcommand{\resumeItemListStart}{\begin{itemize}}
\newcommand{\resumeItemListEnd}{\end{itemize}\vspace{-5pt}}


%-------------------------------------------
%%%%%%  RESUME STARTS HERE  %%%%%%%%%%%%%%%%%%%%%%%%%%%%


\begin{document}

\begin{center}
    \textbf{\Huge \scshape Третьяков Даниил} \\ \vspace{5pt}
    \small +7 900 656 0859 $|$ \href{mailto:trxxxxkov@gmail.com}{trxxxxkov@gmail.com} $|$
    \href{https://t.me/trxxxxkov}{\faTelegram \ trxxxxkov} $|$  \href{https://github.com/trxxxxkov}{\faGithub \ trxxxxkov}
\end{center}


%-----------EDUCATION-----------
\section{Высшее образование}
  \resumeSubHeadingListStart
    \resumeSubheading
      {Санкт-Петербургский Государственный Университет}{2020 г. -- н. в.}
      {Бакалавриат, ПМИ, кафедра математической теории игр и статистических решений}
      {4-й курс}
  \resumeSubHeadingListEnd


%-----------PROGRAMMING SKILLS-----------
\section{Технические навыки}
 \begin{itemize}[leftmargin=0.15in, label={}]
    \small{\item{
        \textbf{Математика}{: эконометрика, теория вероятностей, математическая статистика, линейная алгебра, математический анализ, дискретная математика} \\    
        \vspace{3pt}
        \textbf{Языки программирования}{: Python, C++, SQL (Postgres), R, Bash} \\
        \vspace{3pt}
        \textbf{Фреймворки}{:PyTorch, Scikit-Learn, MLFlow, Apache Hadoop, Apache Spark, Apache Airflow, Aiogram} \\
        \vspace{3pt}
        \textbf{Библиотеки}{: pandas, NumPy, SciPy, Matplotlib, statsmodels} \\
        \vspace{3pt}
        \textbf{Прочее}{: Linux (Arch, btw), Git, Docker, Nginx, SQLAlchemy, BI-системы (DataLens), \href{https://leetcode.com/u/trxxxxkov/}{LeetCode}} \\
    }}
 \end{itemize}

%-----------COMMERCIAL PROJECTS-----------
\section{Коммерческие проекты}
    \resumeSubHeadingListStart
      \resumeProjectHeading
          {\href{https://github.com/trxxxxkov/chxxxxbot.git}{\textbf{Telegram-бот на базе моделей ИИ}} $|$ \emph{Python, Aiogram, PostgreSQL, Docker, Nginx, DataLens}}{}
          \resumeItemListStart
            \resumeItem{Используя \emph{Aiogram} и \emph{OpenAI API},  создал веб-приложение, позволяющее клиентам взаимодействовать с передовыми генеративными моделями на базе ИИ;}
            \resumeItem{Настроил \emph{Yandex DataLens} для агрегации, визуализации и анализа пользовательской статистики с целью проведения статистических экспериментов и повышения качества сервиса;}
            \resumeItem{Интегрировал базу данных \emph{PostgreSQL}, значительно повысив надёжность и масштабируемость проекта;}
            \resumeItem{Используя \emph{Docker Compose}, спроектировал и внедрил микросервисную архитектуру, снизив время развёртывания проекта на новом сервере до 1 минуты;}
            \resumeItem{Настроил \emph{Nginx} как reverse proxy для приёма событий через webhook, что удвоило скорость их обработки;}
          \resumeItemListEnd
    \resumeSubHeadingListEnd


%-----------NON-PROFIT PROJECTS-----------
\section{Некоммерческие проекты}

    \resumeSubHeadingListStart
    
      \resumeProjectHeading
          {\href{https://github.com/trxxxxkov/approxer.git}{\textbf{Генетический алгоритм для сужения и аппроксимация множества Парето}} $|$ \emph{Python, SciPy, Matplotlib}}{}
          \resumeItemListStart
            \resumeItem{Программно реализовал сужение множества Парето с использованием квантов информации от ЛПР;}
            \resumeItem{Проанализировал и сравнил существующие методы аппроксимации множества Парето;}
            \resumeItem{Адаптировал генетический алгоритм для использования в итерационном процессе сужения;}
          \resumeItemListEnd
    \resumeSubHeadingListEnd


%-----------HACKATONS AND CHALLENGES-----------
\section{Участие в соревнованиях}
    \resumeSubHeadingListStart
        \resumeProjectHeading
            {\textbf{Участник соревнования \href{https://backdropbuild.com/}{Backdrop Build в категории "AI"} с проектом \href{https://github.com/trxxxxkov/chxxxxbot.git}{chxxxxbot}.}}{\emph{Июль 2024}}
        \resumeProjectHeading
            {\textbf{Участник хакатона \href{https://e-cup-ozon.ru/\#about}{E-CUP 2024 от Ozon Tech} в команде на позиции Data Engineer.}}{\emph{Август 2024}}
    
    \resumeSubHeadingListEnd


%-----------WORKING CONDITIONS-----------
\section{Предпочтительные условия работы}
\begin{itemize}[leftmargin=0.15in, label={}, itemsep=-0.1em]
    \small{
        \item \textbf{График работы}: 
        \vspace{-8pt}
        \begin{itemize}
            \item Фиксированный график: 20-32 ч. в неделю
            \vspace{-3pt}
            \item Гибкий график: до 40 ч. в неделю
        \end{itemize}
        \vspace{-4pt}
        \item \textbf{Местоположение}: 
        \vspace{-8pt}
        \begin{itemize}
            \item Санкт-Петербург
            \vspace{-3pt}
            \item Удаленная работа
        \end{itemize}
    }
\end{itemize}


%-----------ADDITIONAL SKILLS AND KNOWLEDGE-----------
\section{Дополнительные знания и навыки}
    \resumeSubHeadingListStart
        \resumeProjectHeading
            {Уровень владения английским языком: \textit{Upper Intermediate (B2)} $|$ \emph{\href{https://guestbook.spbu.ru/poslednie-obrashcheniya/9-prorektory-spbgu/babelyuk-ekaterina-gennadevna/7781-status-sertifikata-podtverzhdayushchego-uroven-b2-po-anglijskomu-yazyku.html}{Сертификат СПбГУ}}}{\emph{Июль 2022}}    
        \resumeProjectHeading
            {Математическая статистика $|$ \emph{\href{https://stepik.org/cert/2197375}{Сертификат Computer Science Center}}}{\emph{Октябрь 2023}}
        \resumeProjectHeading
            {Математический анализ $|$ \emph{\href{https://stepik.org/cert/2202369}{Сертификат Computer Science Center}}}{\emph{Октябрь 2023}}
        \resumeProjectHeading
            {Линейная алгебра $|$ \emph{\href{https://stepik.org/cert/2378334}{Сертификат Computer Science Center}}}{\emph{Март 2024}}
        \resumeProjectHeading
            {Теория вероятностей $|$ \emph{\href{https://stepik.org/cert/2536313}{Сертификат Computer Science Center}}}{\emph{Август 2024}}
        \resumeProjectHeading
            {Введение в соревновательный Data Science $|$ \emph{\href{https://stepik.org/cert/2197375}{Сертификат Stepik}}}{\emph{Август 2024}}
    \resumeSubHeadingListEnd

%-------------------------------------------
\end{document}

